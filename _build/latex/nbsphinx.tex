%% Generated by Sphinx.
\def\sphinxdocclass{report}
\documentclass[letterpaper,10pt,english]{sphinxmanual}
\ifdefined\pdfpxdimen
   \let\sphinxpxdimen\pdfpxdimen\else\newdimen\sphinxpxdimen
\fi \sphinxpxdimen=.75bp\relax

\PassOptionsToPackage{warn}{textcomp}
\usepackage[utf8]{inputenc}
\ifdefined\DeclareUnicodeCharacter
% support both utf8 and utf8x syntaxes
  \ifdefined\DeclareUnicodeCharacterAsOptional
    \def\sphinxDUC#1{\DeclareUnicodeCharacter{"#1}}
  \else
    \let\sphinxDUC\DeclareUnicodeCharacter
  \fi
  \sphinxDUC{00A0}{\nobreakspace}
  \sphinxDUC{2500}{\sphinxunichar{2500}}
  \sphinxDUC{2502}{\sphinxunichar{2502}}
  \sphinxDUC{2514}{\sphinxunichar{2514}}
  \sphinxDUC{251C}{\sphinxunichar{251C}}
  \sphinxDUC{2572}{\textbackslash}
\fi
\usepackage{cmap}
\usepackage[T1]{fontenc}
\usepackage{amsmath,amssymb,amstext}
\usepackage{babel}



\usepackage{times}
\expandafter\ifx\csname T@LGR\endcsname\relax
\else
% LGR was declared as font encoding
  \substitutefont{LGR}{\rmdefault}{cmr}
  \substitutefont{LGR}{\sfdefault}{cmss}
  \substitutefont{LGR}{\ttdefault}{cmtt}
\fi
\expandafter\ifx\csname T@X2\endcsname\relax
  \expandafter\ifx\csname T@T2A\endcsname\relax
  \else
  % T2A was declared as font encoding
    \substitutefont{T2A}{\rmdefault}{cmr}
    \substitutefont{T2A}{\sfdefault}{cmss}
    \substitutefont{T2A}{\ttdefault}{cmtt}
  \fi
\else
% X2 was declared as font encoding
  \substitutefont{X2}{\rmdefault}{cmr}
  \substitutefont{X2}{\sfdefault}{cmss}
  \substitutefont{X2}{\ttdefault}{cmtt}
\fi


\usepackage[Bjarne]{fncychap}
\usepackage{sphinx}

\fvset{fontsize=\small}
\usepackage{geometry}


% Include hyperref last.
\usepackage{hyperref}
% Fix anchor placement for figures with captions.
\usepackage{hypcap}% it must be loaded after hyperref.
% Set up styles of URL: it should be placed after hyperref.
\urlstyle{same}

\usepackage{sphinxmessages}
\setcounter{tocdepth}{1}


% Jupyter Notebook code cell colors
\definecolor{nbsphinxin}{HTML}{307FC1}
\definecolor{nbsphinxout}{HTML}{BF5B3D}
\definecolor{nbsphinx-code-bg}{HTML}{F5F5F5}
\definecolor{nbsphinx-code-border}{HTML}{E0E0E0}
\definecolor{nbsphinx-stderr}{HTML}{FFDDDD}
% ANSI colors for output streams and traceback highlighting
\definecolor{ansi-black}{HTML}{3E424D}
\definecolor{ansi-black-intense}{HTML}{282C36}
\definecolor{ansi-red}{HTML}{E75C58}
\definecolor{ansi-red-intense}{HTML}{B22B31}
\definecolor{ansi-green}{HTML}{00A250}
\definecolor{ansi-green-intense}{HTML}{007427}
\definecolor{ansi-yellow}{HTML}{DDB62B}
\definecolor{ansi-yellow-intense}{HTML}{B27D12}
\definecolor{ansi-blue}{HTML}{208FFB}
\definecolor{ansi-blue-intense}{HTML}{0065CA}
\definecolor{ansi-magenta}{HTML}{D160C4}
\definecolor{ansi-magenta-intense}{HTML}{A03196}
\definecolor{ansi-cyan}{HTML}{60C6C8}
\definecolor{ansi-cyan-intense}{HTML}{258F8F}
\definecolor{ansi-white}{HTML}{C5C1B4}
\definecolor{ansi-white-intense}{HTML}{A1A6B2}
\definecolor{ansi-default-inverse-fg}{HTML}{FFFFFF}
\definecolor{ansi-default-inverse-bg}{HTML}{000000}

% Define an environment for non-plain-text code cell outputs (e.g. images)
\makeatletter
\newenvironment{nbsphinxfancyoutput}{%
    % Avoid fatal error with framed.sty if graphics too long to fit on one page
    \let\sphinxincludegraphics\nbsphinxincludegraphics
    \nbsphinx@image@maxheight\textheight
    \advance\nbsphinx@image@maxheight -2\fboxsep   % default \fboxsep 3pt
    \advance\nbsphinx@image@maxheight -2\fboxrule  % default \fboxrule 0.4pt
    \advance\nbsphinx@image@maxheight -\baselineskip
\def\nbsphinxfcolorbox{\spx@fcolorbox{nbsphinx-code-border}{white}}%
\def\FrameCommand{\nbsphinxfcolorbox\nbsphinxfancyaddprompt\@empty}%
\def\FirstFrameCommand{\nbsphinxfcolorbox\nbsphinxfancyaddprompt\sphinxVerbatim@Continues}%
\def\MidFrameCommand{\nbsphinxfcolorbox\sphinxVerbatim@Continued\sphinxVerbatim@Continues}%
\def\LastFrameCommand{\nbsphinxfcolorbox\sphinxVerbatim@Continued\@empty}%
\MakeFramed{\advance\hsize-\width\@totalleftmargin\z@\linewidth\hsize\@setminipage}%
\lineskip=1ex\lineskiplimit=1ex\raggedright%
}{\par\unskip\@minipagefalse\endMakeFramed}
\makeatother
\newbox\nbsphinxpromptbox
\def\nbsphinxfancyaddprompt{\ifvoid\nbsphinxpromptbox\else
    \kern\fboxrule\kern\fboxsep
    \copy\nbsphinxpromptbox
    \kern-\ht\nbsphinxpromptbox\kern-\dp\nbsphinxpromptbox
    \kern-\fboxsep\kern-\fboxrule\nointerlineskip
    \fi}
\newlength\nbsphinxcodecellspacing
\setlength{\nbsphinxcodecellspacing}{0pt}

% Define support macros for attaching opening and closing lines to notebooks
\newsavebox\nbsphinxbox
\makeatletter
\newcommand{\nbsphinxstartnotebook}[1]{%
    \par
    % measure needed space
    \setbox\nbsphinxbox\vtop{{#1\par}}
    % reserve some space at bottom of page, else start new page
    \needspace{\dimexpr2.5\baselineskip+\ht\nbsphinxbox+\dp\nbsphinxbox}
    % mimick vertical spacing from \section command
      \addpenalty\@secpenalty
      \@tempskipa 3.5ex \@plus 1ex \@minus .2ex\relax
      \addvspace\@tempskipa
      {\Large\@tempskipa\baselineskip
             \advance\@tempskipa-\prevdepth
             \advance\@tempskipa-\ht\nbsphinxbox
             \ifdim\@tempskipa>\z@
               \vskip \@tempskipa
             \fi}
    \unvbox\nbsphinxbox
    % if notebook starts with a \section, prevent it from adding extra space
    \@nobreaktrue\everypar{\@nobreakfalse\everypar{}}%
    % compensate the parskip which will get inserted by next paragraph
    \nobreak\vskip-\parskip
    % do not break here
    \nobreak
}% end of \nbsphinxstartnotebook

\newcommand{\nbsphinxstopnotebook}[1]{%
    \par
    % measure needed space
    \setbox\nbsphinxbox\vbox{{#1\par}}
    \nobreak % it updates page totals
    \dimen@\pagegoal
    \advance\dimen@-\pagetotal \advance\dimen@-\pagedepth
    \advance\dimen@-\ht\nbsphinxbox \advance\dimen@-\dp\nbsphinxbox
    \ifdim\dimen@<\z@
      % little space left
      \unvbox\nbsphinxbox
      \kern-.8\baselineskip
      \nobreak\vskip\z@\@plus1fil
      \penalty100
      \vskip\z@\@plus-1fil
      \kern.8\baselineskip
    \else
      \unvbox\nbsphinxbox
    \fi
}% end of \nbsphinxstopnotebook

% Ensure height of an included graphics fits in nbsphinxfancyoutput frame
\newdimen\nbsphinx@image@maxheight % set in nbsphinxfancyoutput environment
\newcommand*{\nbsphinxincludegraphics}[2][]{%
    \gdef\spx@includegraphics@options{#1}%
    \setbox\spx@image@box\hbox{\includegraphics[#1,draft]{#2}}%
    \in@false
    \ifdim \wd\spx@image@box>\linewidth
      \g@addto@macro\spx@includegraphics@options{,width=\linewidth}%
      \in@true
    \fi
    % no rotation, no need to worry about depth
    \ifdim \ht\spx@image@box>\nbsphinx@image@maxheight
      \g@addto@macro\spx@includegraphics@options{,height=\nbsphinx@image@maxheight}%
      \in@true
    \fi
    \ifin@
      \g@addto@macro\spx@includegraphics@options{,keepaspectratio}%
    \fi
    \setbox\spx@image@box\box\voidb@x % clear memory
    \expandafter\includegraphics\expandafter[\spx@includegraphics@options]{#2}%
}% end of "\MakeFrame"-safe variant of \sphinxincludegraphics
\makeatother

\makeatletter
\renewcommand*\sphinx@verbatim@nolig@list{\do\'\do\`}
\begingroup
\catcode`'=\active
\let\nbsphinx@noligs\@noligs
\g@addto@macro\nbsphinx@noligs{\let'\PYGZsq}
\endgroup
\makeatother
\renewcommand*\sphinxbreaksbeforeactivelist{\do\<\do\"\do\'}
\renewcommand*\sphinxbreaksafteractivelist{\do\.\do\,\do\:\do\;\do\?\do\!\do\/\do\>\do\-}
\makeatletter
\fvset{codes*=\sphinxbreaksattexescapedchars\do\^\^\let\@noligs\nbsphinx@noligs}
\makeatother



\title{Particle physics}
\date{Mar 13, 2020}
\release{}
\author{}
\newcommand{\sphinxlogo}{\vbox{}}
\renewcommand{\releasename}{}
\makeindex
\begin{document}

\pagestyle{empty}
\sphinxmaketitle
\pagestyle{plain}
\sphinxtableofcontents
\pagestyle{normal}
\phantomsection\label{\detokenize{index::doc}}


Content generated from the OpenLearn Unit \sphinxhref{https://www.open.edu/openlearn/science-maths-technology/particle-physics/content-section-0}{Particle physics}.


\chapter{Contents:}
\label{\detokenize{index:contents}}

\section{Session 00}
\label{\detokenize{index:session-00}}

\subsection{1  Particle physics in context}
\label{\detokenize{content/session_00/Part_00_01:1_xa0_xa0Particle-physics-in-context}}\label{\detokenize{content/session_00/Part_00_01::doc}}
Although particle physics is often thought of as a ‘new’ concept, it has in fact been around for much, much longer. Like many of the great fundamental theories, the concept of an atomic like structure was first proposed by the ancient Greeks, by the philosopher Democritus (460\textendash{}370BC). He believed that the matter we see around us was formed from a variety of different atoms. While this is known to be incorrect (and if you didn’t know this, you will soon!) in two ways (there are only 92 stable
atoms, not a wide variety and they are divisible into smaller components), it was a great place to start from.

Fast forward a few thousand years, and by the 1930s, it was recognised that atomic nuclei are composed of \sphinxstylestrong{protons} and \sphinxstylestrong{neutrons}, and that in atoms the nuclei are surrounded by clouds of \sphinxstylestrong{electrons}. The existence of one more type of particle was surmised \sphinxhyphen{} the \sphinxstylestrong{electron neutrino} \sphinxhyphen{} which is produced in some radioactive decays when atoms change from one type to another. Finally, the \sphinxstylestrong{photon} was recognised as the quantum of energy involved in electromagnetic interactions.

Eighty years ago the account of subatomic structure would have ended with these particles, but now a deeper layer of structure is known. It is believed that protons and neutrons are composed of structureless particles known as quarks, while \sphinxstylestrong{electrons} and \sphinxstylestrong{electron neutrinos} are merely examples of a broader class of particles known as leptons. Furthermore, there are other fundamental interactions, besides electromagnetism, each of which has its own set of quanta. Leptons and quarks are
discussed in the following sections, and the quanta of other fundamental interactions are discussed at the end of this course.

Before embarking on this journey through particle physics, consider the following puzzle: all of the atoms that make up the Universe (apart from hydrogen atoms) contain neutrons as well as protons in their nuclei. Yet free neutrons (that is, those not inside nuclei) undergo beta decay with a half\sphinxhyphen{}life of about 10 minutes. In the early Universe, soon after the Big Bang and before atoms formed, there are believed to have existed equal numbers of protons and neutrons. So why didn’t the neutrons all
decay at that time, leaving a Universe made only of hydrogen? You will return to answer this puzzle at the end of this course.


\subsection{2  Leptons}
\label{\detokenize{content/session_00/Part_00_02:2_xa0_xa0Leptons}}\label{\detokenize{content/session_00/Part_00_02::doc}}
As it stands, the smallest known building blocks of matter showing no evidence of comprising of smaller or simpler constituents can be divided into two groups: leptons and quarks. First, let us tackle the former. There are six varieties of leptons. Perhaps the most familiar of these being electrons (e−) and electron neutrinos (νe).

In 1936 and 1975, two more particles, with the same charge as the electron, only rather heavier, were discovered. The first is known as the \sphinxstylestrong{muon} (represented by μ−; the Greek letter \sphinxstyleemphasis{mu}), which is about 200 times heavier than the electron. The second is called the \sphinxstylestrong{tauon} (represented by τ−; the Greek letter \sphinxstyleemphasis{tau}), which is about 3500 times heavier than the electron. The superscript minus signs on the electron, muon and tauon indicate that these particles have negative electric charge.
Muons and tauons are unstable, and rapidly decay into electrons and neutrinos in a fraction of a second.

Like the electron, the muon and tauon each have an associated neutrino: the \sphinxstylestrong{muon neutrino}(νμ) and the \sphinxstylestrong{tauon neutrino}(ντ), each with zero electric charge. Particle physicists have not yet been able to measure the masses of neutrinos; all that is known is that the three types of neutrino have a combined mass that is several million times less than that of an electron, but not zero.

These six fundamental particles are collectively referred to as leptons. (The word lepton comes from the Greek \sphinxstyleemphasis{leptos}, meaning ‘thin’ or ‘lightweight’.) The six different types are often referred to as different \sphinxstylestrong{flavours} of lepton, and the three pairs of particles are often referred to as three \sphinxstylestrong{generations} of leptons.





\sphinxstylestrong{Table 1} Three generations of leptons













First generation





Second generation





Third generation









Leptons with a charge \sphinxstyleemphasis{−e}





\sphinxincludegraphics[width=31\sphinxpxdimen,height=31\sphinxpxdimen]{{sm123_t06_p03_e_b.eps}.jpg}





\sphinxincludegraphics[width=32\sphinxpxdimen,height=32\sphinxpxdimen]{{sm123_t06_p03_mu_b.eps}.jpg}





\sphinxincludegraphics[width=31\sphinxpxdimen,height=31\sphinxpxdimen]{{sm123_t06_p03_tau_b.eps}.jpg}









Leptons with charge 0





\sphinxincludegraphics[width=33\sphinxpxdimen,height=31\sphinxpxdimen]{{sm123_t06_p03_nu_b.eps}.png}





\sphinxincludegraphics[width=32\sphinxpxdimen,height=32\sphinxpxdimen]{{sm123_t06_p03_nu_mu.eps}.jpg}





\sphinxincludegraphics[width=32\sphinxpxdimen,height=32\sphinxpxdimen]{{sm123_t06_p03_nu_tau.eps}.jpg}









To each lepton there corresponds an antilepton with opposite charge (if charged) but with the same mass. These \sphinxstylestrong{antimatter} particles are denoted by the symbols e+, μ+ and τ+ for the charged leptons and ν¯e⁢, ν¯μ and ν¯τ for the neutral leptons. The antielectron is often referred to as a \sphinxstylestrong{positron}.


\subsection{3  Quarks}
\label{\detokenize{content/session_00/Part_00_03:3_xa0_xa0Quarks}}\label{\detokenize{content/session_00/Part_00_03::doc}}
The other group of fundamental particles which was introduced in the previous section are the quarks (nope, it’s not just a tasty cheese!). In 1964, two independent researchers were working on theories to explain strong interaction symmetry in particle physics. These were Murray Gell\sphinxhyphen{}Mann and George Zweig. These scientists proposed that the properties of hadrons (more on these in the next section) could be explained if they were composed of smaller constituents. What is known now to be quarks!

\sphinxincludegraphics[width=512\sphinxpxdimen,height=351\sphinxpxdimen]{{sm123_1_figure1_murray_george.tif}.jpg}

Figure 1 Murray Gell\sphinxhyphen{}Mann (left) and George Zweig (right).

Like the leptons, there are also, handily, 6 varieties of these. They are known as the six flavours of quarks, labelled by the letters u, d, c, s, t and b, which stand for \sphinxstylestrong{up}, \sphinxstylestrong{down}, \sphinxstylestrong{charm}, \sphinxstylestrong{strange}, \sphinxstylestrong{top} and \sphinxstylestrong{bottom}. The up, charm and top quarks each carry a positive charge of +23e, while the down, strange and bottom quarks each carry a negative charge of −13e.

Like the leptons, the six quarks are often grouped into three generations on the basis of their mass, with the first generation being the least massive. To each quark, there corresponds an antiquark, with the opposite charge and the same mass. These are denoted by the symbols u¯, d¯, c¯, s¯, t¯ and b¯. So anti\sphinxhyphen{}up, anti\sphinxhyphen{}charm and anti\sphinxhyphen{}top antiquarks each carry a negative charge of −23e, while anti\sphinxhyphen{}down, anti\sphinxhyphen{}strange and anti\sphinxhyphen{}bottom antiquarks each carry a positive charge of +13.





\sphinxstylestrong{Table 2} Three generations of quarks













1st generation





2nd generation





3rd generation









Quarks with charge +23e





\sphinxincludegraphics[width=32\sphinxpxdimen,height=32\sphinxpxdimen]{{sm123_t06_p03_u.eps}.jpg}





\sphinxincludegraphics[width=32\sphinxpxdimen,height=32\sphinxpxdimen]{{sm123_t06_p03_c.eps}.jpg}





\sphinxincludegraphics[width=31\sphinxpxdimen,height=31\sphinxpxdimen]{{sm123_t06_p03_t.eps}.jpg}









Quarks with charge −13e





\sphinxincludegraphics[width=32\sphinxpxdimen,height=32\sphinxpxdimen]{{sm123_t06_p03_d.eps}.jpg}





\sphinxincludegraphics[width=32\sphinxpxdimen,height=32\sphinxpxdimen]{{sm123_t06_p03_s.eps}.jpg}





\sphinxincludegraphics[width=33\sphinxpxdimen,height=33\sphinxpxdimen]{{sm123_t06_p03_b.eps}.jpg}









Of all the quarks, the up and down are the least massive of the flavours. The charm and strange quarks and antiquarks are more massive than the up and down quarks, and the top and bottom quarks and antiquarks are yet more massive still.


\subsection{4  Hadrons}
\label{\detokenize{content/session_00/Part_00_04:4_xa0_xa0Hadrons}}\label{\detokenize{content/session_00/Part_00_04::doc}}
As well as the leptons and quarks, there is another quite different group in the mix \sphinxhyphen{} hadrons! Perhaps the most familiar and even intuitive class of particles. A hadron is a particle which is composed of two of more quarks, a composite particle of which protons and neutrons are examples of. The quarks are held together by the ‘strong force’ (yes, that is what it is called, for hopefully obvious reasons!), much in the same way that the electromagnetic force holds molecules together. This helps
explain the name\sphinxhyphen{} the word hadron comes from the Greek \sphinxstyleemphasis{hadros}, meaning ‘strong’ or ‘robust’. This is a good point to briefly introduce the fundamental forces found in nature. In everyday parlance, the word fundamental is often used with reckless abandon, but it has a very particular, important meaning here. These forces are named as such because they cannot be explained due to the action of another force. For example, friction is NOT a fundamental force as it can be explained as occurring as a
result of the interaction of the electromagnetic forces in one atom with those in another. Table 3 shows a summary of their key properties and they are listed in order of strength with the strongest at the top. Notice how weak gravity really is! And it’s just as well, it would be awfully difficult to lift your foot up if it wasn’t!





Table 3 Key properties









\sphinxstylestrong{Force}





\sphinxstylestrong{Experienced by}





\sphinxstylestrong{Range (m)}(distance between 2 objects before force becomes negligible)





\sphinxstylestrong{Strength}(relative to EM force)









Strong





Quarks





10\sphinxhyphen{}15





100









Electromagnetic





Charged particles





\(\infty\)





1









Weak





Any particle





10\sphinxhyphen{}17





10\sphinxhyphen{}5









Gravitation





Any particle with a mass





\(\infty\)





10\sphinxhyphen{}38









Although the only hadrons existing in the everyday world are protons and neutrons, many more types of hadron can be created in high\sphinxhyphen{}energy particle collisions. Such reactions are common in the upper atmosphere where high\sphinxhyphen{}energy protons from outer space (known as cosmic rays) collide with nuclei of nitrogen and oxygen, smashing them apart and creating new hadrons. Since the 1960s, such reactions have been closely studied in laboratories such as CERN (the European Organization for Nuclear
Research), where high\sphinxhyphen{}energy beams of particles are smashed together.

Although many dozens of different types of hadron may be created in this way, all of the new ones are unstable and rapidly decay into other, long\sphinxhyphen{}lived particles, such as leptons, protons and neutrons. Fortunately, it’s not necessary to dwell on (let alone remember) the names and properties of all the types of hadron, because there is a straightforward description for building them from particles that are believed to be fundamental, namely from quarks and antiquarks.


\subsubsection{4.1 Building a hadron}
\label{\detokenize{content/session_00/Part_00_04:4.1-Building-a-hadron}}
Quarks and antiquarks only occur bound together inside hadrons; they have never been observed in isolation. While it may seem like the process of ‘building’ a hadron would be something very complicated (and yes, the underlying physics is indeed complex), there are in fact 3 simple ‘recipes’ to remember.

A hadron can consist of either:
\begin{itemize}
\item {} 
Three quarks (in which case it is called a \sphinxstylestrong{baryon}).

\item {} 
Three antiquarks (in which case it is called an \sphinxstylestrong{antibaryon}).

\item {} 
One quark and one antiquark (in which case it is called a \sphinxstylestrong{meson}).

\end{itemize}

\sphinxincludegraphics[width=187\sphinxpxdimen,height=528\sphinxpxdimen]{{sm123_t06_p03_f01.eps}.jpg}

\sphinxstylestrong{Figure 2} The three recipes for building hadrons from quarks. Quarks and antiquarks with a charge of two\sphinxhyphen{}thirds that of a proton or electron are shown in purple, and those with a charge of one\sphinxhyphen{}third that of a proton or electron are shown in orange. The symbol q represents a quark, and q¯ represents an antiquark. Possible combinations of quarks making (a) a baryon, (b) an antibaryon, and (c) a meson are shown.


\paragraph{Activity 1 The quark fruit machine}
\label{\detokenize{content/session_00/Part_00_04:Activity-1_xa0The-quark-fruit-machine}}
\sphinxstylestrong{Timing: Allow approximately 30 minutes}


\subparagraph{Question}
\label{\detokenize{content/session_00/Part_00_04:Question}}
In this activity, you will play on a ‘fruit machine’ to investigate the combinations of quarks and antiquarks that make up different baryons, antibaryons and mesons. Read the introduction and then click on the tabs to answer questions relating to each type of hadron within the fruit machine. Finish by reading the summary.

Any combination of quarks and antiquarks that obeys one of the three recipes above is a valid hadron. The net electric charge of a hadron is simply the sum of the electric charges of the quarks or antiquarks of which it is composed. As you saw in the activity, the net charge of a hadron is therefore always a whole number, despite the fact that the quarks themselves have non\sphinxhyphen{}whole number electric charge.

As a specific example of the hadron\sphinxhyphen{}building recipe, the proton is a baryon, so it is composed of three quarks, and as mentioned above, it is composed of up and down quarks only.

The proton has a charge of +e. The only way that three up or down quarks can be combined to make this net charge is by combining two up quarks with a down quark. So the quark content of a proton is (uud), giving a net charge of +2e3+2e3−e3=+e.


\subparagraph{Question}
\label{\detokenize{content/session_00/Part_00_04:id1}}
What is the antiquark composition and charge of an antiproton?


\subparagraph{Answer}
\label{\detokenize{content/session_00/Part_00_04:Answer}}
An antiproton has a similar composition to a proton but is composed of antiquarks rather than quarks. Its composition is therefore (uud¯) giving a net charge of −2e3−2e3+e3=−e.


\subparagraph{Question}
\label{\detokenize{content/session_00/Part_00_04:id2}}
What combination of three up or down quarks would make a neutron?


\subparagraph{Answer}
\label{\detokenize{content/session_00/Part_00_04:id3}}
A neutron has zero charge and is composed of three up or down quarks, so its quark content must be (udd), giving a net charge of +2e3−e3−e3=0.


\subparagraph{Question}
\label{\detokenize{content/session_00/Part_00_04:id4}}
Pions are the least massive examples of mesons. They are composed of only up or down quarks and antiquarks. What do you suppose are the compositions of the following pions: π+, π−, π0? (The superscript indicates the electric charge in each case).


\subparagraph{Answer}
\label{\detokenize{content/session_00/Part_00_04:id5}}\begin{itemize}
\item {} 
π+ must be composed of (ud¯) giving a net charge of +2e3+e3=+e.

\item {} 
π− must be composed of (du¯) giving a net charge of −e3−2e3=−e.

\item {} 
π0 could be composed of (uu¯) or (dd¯) giving a net charge of +2e3−2e3=0 or −e3+e3=0. (In fact neutral pions exist with \sphinxstyleemphasis{either} composition.) The tally of six leptons and six quarks, each with their own antiparticles, may seem like a huge number of fundamental particles, but don’t let this put you off. Virtually everything in the Universe is made up of merely the first generation of each type of particle (see Tables 1 and 2), namely:

\item {} 
electrons

\item {} 
up quarks, and

\item {} 
down quarks,

\end{itemize}

with electron neutrinos being created in radioactive decays.

As for the other generations:
\begin{itemize}
\item {} 
the second generation of leptons (muon and muon neutrino)

\item {} 
the second generation of quarks (charm and strange)

\item {} 
the third generation of leptons (tauon and tauon neutrino), and

\item {} 
the third generation of quarks (top and bottom)

\end{itemize}

all have exactly the same properties as their first\sphinxhyphen{}generation counterparts except that they are more massive.

Quite why nature decided to repeat this invention three times over is not currently understood!


\subsection{5  High\sphinxhyphen{}energy reactions}
\label{\detokenize{content/session_00/Part_00_05:5_xa0_xa0High-energy-reactions}}\label{\detokenize{content/session_00/Part_00_05::doc}}
In particle accelerators, or in the Earth’s upper atmosphere, hadrons can smash into each other with large kinetic energies and create new hadrons from the debris left behind. The following short video was made while the ATLAS detector at CERN was being built. This detector subsequently discovered evidence for the Higgs boson in 2012.

\sphinxstylestrong{Video 1} Building the ATLAS detector.









\sphinxstylestrong{NARRATOR::} \sphinxstyleemphasis{ATLAS is one of the biggest and most complicated machines ever built. The size of it is overwhelming, 48 metres long, 25 metres in diameter, in a cavern the size of a cathedral, and the whole thing built 100 metres underground. ATLAS sits in the 27 kilometre ring at CERN and it’s here that protons at nearly the speed of light smash into one another.}; \sphinxstyleemphasis{ATLAS’s job is to detect fragments exploding out of the resulting mini Big Bang. One of the scientists building this vast machine
is Pippa Wells.};

\sphinxstylestrong{PIPPA WELLS::} \sphinxstyleemphasis{If you can see, there’s a hole there…};

\sphinxstylestrong{NARRATOR: :} \sphinxstyleemphasis{In the middle.};

\sphinxstylestrong{PIPPA WELLS: :} \sphinxstyleemphasis{…in the middle of that disc. That is where the beam pipe from the accelerator will finally go.};

\sphinxstylestrong{NARRATOR::} \sphinxstyleemphasis{So the protons are going to be whistling in and through and out of that hole in the middle?};

\sphinxstylestrong{PIPPA WELLS::} \sphinxstyleemphasis{That’s right. So bunches coming in from this side, same thing coming in from the other side. Forty million packets of protons every second have a chance to collide in the middle of the detector. And something like 20 of those protons will interact. So we’re looking at the remnants of what was briefly created in the middle, and then those remnants fly out in all directions.};

\sphinxstylestrong{NARRATOR::} \sphinxstyleemphasis{So you’re saying two protons hit, there’s a fireball, a massive, or minute, explosion.};

\sphinxstylestrong{PIPPA WELLS::} \sphinxstyleemphasis{A tiny fireball, yes.};

\sphinxstylestrong{NARRATOR::} \sphinxstyleemphasis{But that’d be incredibly hot.};

\sphinxstylestrong{PIPPA WELLS: :} \sphinxstyleemphasis{Yes.};

\sphinxstylestrong{NARRATOR: :} \sphinxstyleemphasis{And whatever is formed will then instantly break down into other bits.};

\sphinxstylestrong{PIPPA WELLS: :} \sphinxstyleemphasis{That’s right.};

\sphinxstylestrong{NARRATOR: :} \sphinxstyleemphasis{ATLAS is so huge because it’s made of layer upon layer of detector designed to work out what went where, and when. Pippa is responsible for the detectors at the very centre of ATLAS, closest to the collision. But detecting the particles is just the beginning. The next challenge is getting the information out, down 50,000 separate cables.};

\sphinxstylestrong{PIPPA WELLS: :} \sphinxstyleemphasis{What we then have to do is work out, was this an interesting event? Are these the remnants of some new particle we’ve never seen before? Is this a clue for the Higgs boson, or for dark matter, or is this just some ordinary interaction? It’s the sort of thing we’ve seen at previous accelerators, it’s just happening now at higher energy.};

\sphinxstylestrong{NARRATOR::} \sphinxstyleemphasis{Right. It won’t be humans deciding which collisions are interesting. ATLAS can handle 600 million collisions a second. Ultrafast computers analyse events as they happen and decide which are worth recording and which can be thrown away.}; \sphinxstyleemphasis{This is a very big bit of apparatus you’ve got here, isn’t it?};

\sphinxstylestrong{PIPPA WELLS::} \sphinxstyleemphasis{It certainly is, yes.};

\sphinxstylestrong{NARRATOR: :} \sphinxstyleemphasis{Are you sure it’s going to work?};

\sphinxstylestrong{PIPPA WELLS: :} \sphinxstyleemphasis{Yes.};









\sphinxincludegraphics[width=512\sphinxpxdimen,height=288\sphinxpxdimen]{{sm123_2018b_vid101-640x360_openlearn}.jpg} Einstein’s most famous equation and in fact arguably the most famous equation in history tells us that mass (\sphinxstyleemphasis{m}) and energy (\sphinxstyleemphasis{E}) are interchangeable via E=m⁢  c2, where \sphinxstyleemphasis{c} is the speed of light, 3.00 × 108 m s\textendash{}1. The energies involved are usually expressed in terms of a unit called the \sphinxstylestrong{electronvolt}. As a result, it’s also convenient to refer to the masses of subatomic particles in terms of their energy equivalence. For instance, the
mass of an electron is about 9.1 × 10\textendash{}31 kg. The energy equivalent of this mass is given by Einstein’s equation: E=mc2=(9.1×10−31)×(3×108)2=(8.19×10−14)kgm2s2 One electronvolt is defined to be 1.602×10−19J

Therefore, E=(8.19×10−14)(1.602×10−19)=511236 eV, over 510 thousand electronvolts, or 510 keV (kilo electronvolts). The term ‘mass energy’ is often used to refer to this energy equivalent of the mass of a particle. Similarly, the mass energy of either a proton or neutron is around 940 MeV (940 mega electronvolts or 940 million electronvolts).

As an example of the high\sphinxhyphen{}energy reactions that can occur inside ATLAS, when a proton with kinetic energy of several hundred MeV collides with another proton or a neutron, new hadrons, such as pions, can be created.

In such processes, some of the kinetic energy of the protons is converted into mass, via the familiar equation E=m⁢  c2, and so appears in the form of new particles. As you have learned, the pions that are created come in three varieties:
\begin{itemize}
\item {} 
π+, with positive charge

\item {} 
π− with negative charge, and

\item {} 
a neutral π0 with zero charge.

\end{itemize}

Their masses are each around 140 MeV/\sphinxstyleemphasis{c}2, so 140 MeV of kinetic energy is required to create each pion.


\subsubsection{Question}
\label{\detokenize{content/session_00/Part_00_05:Question}}
There is a hadron, formed in the collisions of pions and nucleons, with charge +2\sphinxstyleemphasis{e}. How can a single pion combine with a single nucleon to produce nothing but this new hadron? Is it a baryon, an antibaryon or a meson?


\subsubsection{Answer}
\label{\detokenize{content/session_00/Part_00_05:Answer}}
Pions and nucleons contain only up and down quarks, so these are the only ‘raw materials’ available from which to build the new hadron. To get a charge of +2\sphinxstyleemphasis{e} requires three up quarks (+2e3+2e3+2e3=+2e). This hadron can be formed by the collision of π+, (ud¯), with a proton, (uud), followed by the annihilation of d with d¯. As the hadron contains three quarks (uuu), it is a baryon. The outcome from a high\sphinxhyphen{}energy collision between, say, two protons is subject to quantum indeterminacy in
several ways. Imagine that two protons collide with a total kinetic energy of 200 MeV. This is enough energy to make a pion and still have a little energy left over to provide kinetic energy of the products. But what exactly will the products be? In fact there are two possibilities in this case:













p+p\(\rightarrow\)p+p+π0





p+p\(\rightarrow\)p+n+π+









Each of these possibilities will occur, and there is no way of predicting which one will be the outcome of a particular reaction. The rules that must be obeyed in these reactions are:
\begin{itemize}
\item {} 
energy is conserved (as usual)

\item {} 
electric charge is conserved (as usual)

\item {} 
the number of quarks minus the number of antiquarks is conserved.

\end{itemize}

Now look at each of these rules in more detail.

First, look at the conservation of energy, which the rules say must be conserved. There is 200 MeV of available kinetic energy, and 140 MeV is used up in creating the pion in each case. This leaves 60 MeV for kinetic energy of the products. It’s not possible that two pions can be created as that would require at least 2 × 140 MeV = 280 MeV of kinetic energy from the reactants.

The second rule is that electric charge is conserved. In the two proton reactions, the net electric charge of the reactants is twice that of a proton in each case. The net electric charge of the products in the two cases is also twice that of a proton, so electric charge is conserved in both cases.

Finally, the third rule says that the number of quarks minus the number of antiquarks is conserved. In the two proton reactions, the reactants are composed of 3 quarks for each proton and no antiquarks, so the number of quarks minus the number of antiquarks is 6 for the reactants. In the products, the number of quarks is 3 for each proton or neutron and 1 for each pion, while the number of antiquarks is 1 for each pion. The number of quarks minus the number of antiquarks in the products is
therefore (3 + 3 + 1 − 1) = 6.

Therefore, all three rules are indeed obeyed in both reactions.


\subsubsection{Question}
\label{\detokenize{content/session_00/Part_00_05:id1}}
Imagine that two neutrons collide with a combined kinetic energy of 300 MeV. Write down all six possible reactions for producing pions that may occur.


\subsubsection{Answer}
\label{\detokenize{content/session_00/Part_00_05:id2}}
When two neutrons collide with a combined kinetic energy of 300 MeV, this is enough energy to make two pions (with mass energy 140 MeV each). In addition, all the possible reactions to make only one pion can also occur.

In all cases, electric charge must be conserved. As the charge of the two original neutrons is zero, the net charge of the products must also be zero. The total range of possibilities is as follows:













n+n\(\rightarrow\)n+n+π0









n+n\(\rightarrow\)n+p+π−









n+n\(\rightarrow\)n+n+π0+π0









n+n\(\rightarrow\)n+n+π++π−









n+n\(\rightarrow\)n+p+π−+π0









n+n\(\rightarrow\)p+p+π−+π−









The total electric charge on both sides of each reaction listed is zero. In conclusion, just as electrons, protons and neutrons simplify the elements to their essentials, so a new layer of apparent complexity can be understood by the quark model. However, there is a crucial difference. If an atom is hit hard enough, electrons come out. If a nucleus is hit hard enough, nucleons come out. But if a nucleon is hit hard with another nucleon, quarks do not come out. Instead the kinetic energy is
converted into the mass of new hadrons.


\paragraph{Activity 2 Run your own particle accelerator}
\label{\detokenize{content/session_00/Part_00_05:Activity-2_xa0Run-your-own-particle-accelerator}}
\sphinxstylestrong{Timing: Allow approximately 30 minutes}


\subsubsection{Question}
\label{\detokenize{content/session_00/Part_00_05:id3}}
This is an interactive ‘clicker’ game hosted by CERN in which you can run your own particle accelerator laboratory. It was written by four summer students working at CERN, over the course of a 48\sphinxhyphen{}hour hackathon. Click on the following link to launch the game; you may wish to launch it in a new window using ctrl+click.

\sphinxhref{https://cern.ch/particle-clicker}{CERN particle accelerator laboratory ‘clicker’ game}

Start your experiment by repeatedly ‘clicking’ within the particle accelerator to register particle reactions. As you collect more data, research discoveries will appear on the left\sphinxhyphen{}hand side under the ‘Research’ tab, which you can unlock and read about. As you make more discoveries, so the reputation of your laboratory will increase, and the funding for your lab will grow too. As you gain funding, from the ‘HR’ tab you can hire Masters students, PhD students, Postdocs and other more senior
staff who will automatically produce data, without the need for you to click. From the ‘Upgrades’ tab, you can also buy improvements to the equipment and extra support for your staff, which will increase their productivity.

Once you have things running smoothly at your lab, you can leave things running in the background and check back occasionally to see how the research is going. How long does it take you to discover the Higgs boson?

In order to understand how and why high\sphinxhyphen{}energy particle reactions occur, you will now examine the fundamental interaction that is responsible for how quarks behave, known as the strong interaction.


\subsection{6  Strong interactions}
\label{\detokenize{content/session_00/Part_00_06:6_xa0_xa0Strong-interactions}}\label{\detokenize{content/session_00/Part_00_06::doc}}
As introduced in Section 4, the force that binds quarks together inside nucleons (i.e. neutrons and protons) is known as the \sphinxstylestrong{strong interaction} and has a very short range. It operates essentially only within the size of a nucleon.

When two up quarks and a down quark form a proton, or when two down quarks and an up quark form a neutron, the strong interaction has, largely, done its job, in much the same way, for example, that the electric interaction between a proton and an electron does its job by forming a hydrogen atom.

In addition, there is a residual strong interaction between nucleons, which you can imagine as ‘leaking out’ of the individual protons and neutrons. This is sufficient to bind them together in nuclei and is similar in nature to the residual electromagnetic interactions between atoms that are responsible for the formation of molecules.

\sphinxincludegraphics[width=488\sphinxpxdimen,height=256\sphinxpxdimen]{{sm123_t06_p03_f02_kb.eps}.jpg}

\sphinxstylestrong{Figure 3} (a) The strong interaction binds quarks together in nucleons. A residual strong interaction binds nucleons in nuclei. (b) The electrical interaction binds electrons and nuclei together in atoms. A residual electrical interaction binds atoms together in molecules.


\subsubsection{6.1  Gluons}
\label{\detokenize{content/session_00/Part_00_06:6.1_xa0_xa0Gluons}}
It is amazing just how strong the strong interaction between quarks is. At a separation of around 10−15 m \textendash{} the typical size of a proton or neutron \textendash{} the force of attraction between a pair of quarks is equivalent to the weight of a 10\sphinxhyphen{}tonne truck! As might be suggested by its name, the strong force of attraction between two up quarks is much larger than the electric force of repulsion between them. It is this strong force that prevents quarks from being liberated in high\sphinxhyphen{}energy collisions. Free
quarks are never seen to emerge from such processes: quarks only exist confined within baryons or mesons. It is as if they are stuck together with very strong glue.

\sphinxincludegraphics[width=512\sphinxpxdimen,height=512\sphinxpxdimen]{{sm123_t06_p03_f03.tif}.jpg}

\_\_Figure 4 \_\_Schematic image from an old\sphinxhyphen{}style television monitor of the particle tracks resulting from a collision in the Large Electron\textendash{}Positron (LEP) collider at CERN.

A clue concerning the nature of this ‘quark glue’ can be seen in high\sphinxhyphen{}energy experiments such as those conducted at CERN. When high\sphinxhyphen{}energy beams of electrons are collided with equally high\sphinxhyphen{}energy beams of positrons travelling in the opposite direction, it often happens that hadrons emerge as a pair of jets, with each jet made up of a number of hadrons.

The basic interaction that produces the pair of jets is as follows. First, the electron (e−) and positron (e+) annihilate each other and produce what is known as a \sphinxstylestrong{virtual photon}. The reason for this name is that the virtual photon only has a temporary existence and immediately undergoes a pair creation event, giving rise to a quark\textendash{}antiquark pair. Their kinetic energy and mass energy is almost immediately converted into the kinetic energy and mass energy of many more matter and antimatter
particles, including lots more quarks and antiquarks. The many quarks and antiquarks then combine to form a variety of hadrons, and it is only the hadrons that then emerge from the collision as a pair of jets.

\sphinxincludegraphics[width=392\sphinxpxdimen,height=220\sphinxpxdimen]{{sm123_t06_p03_f04.eps}.jpg}

\sphinxstylestrong{Figure 5} An electron and a positron mutually annihilate each other to create a virtual photon which subsequently creates a quark\textendash{}antiquark pair. The energy of the quark and antiquark are then transformed into many more quarks and antiquarks, which give rise to a pair of jets of hadrons.

But this is not the whole story: sometimes \sphinxstyleemphasis{three} jets may be produced. This process involves a particle you haven’t met so far, the \sphinxstylestrong{gluon}. Gluons are the quanta of energy whose emission and absorption are regarded as the origin of strong interactions. They are responsible for ‘gluing’ the quarks strongly together inside hadrons. However, unlike photons \textendash{} but like quarks and antiquarks \textendash{} gluons cannot escape to large distances.

Nonetheless, a quark (or indeed an antiquark) may emit a gluon. The mass energy and kinetic energy of the gluon is quickly turned into the mass energy and kinetic energy of further pairs of quarks and antiquarks. These in turn combine with each other to form various hadrons, and the hadrons produced from the gluon then give rise to a third jet emerging from the process.

\sphinxincludegraphics[width=427\sphinxpxdimen,height=220\sphinxpxdimen]{{sm123_t06_p03_f05.eps}.jpg}

\sphinxstylestrong{Figure 6} An electron and a positron mutually annihilate each other to create a virtual photon which subsequently creates a quark\textendash{}antiquark pair. The quark emits a gluon, shown by the curly line. The energy of the quark, antiquark and gluon are then transformed into many more quarks and antiquarks, which give rise to three jets of hadrons.


\subsubsection{6.2  Quantum chromodynamics}
\label{\detokenize{content/session_00/Part_00_06:6.2_xa0_xa0Quantum-chromodynamics}}
The quantum theory of the strong interactions between quarks and gluons is called \sphinxstylestrong{quantum chromodynamics} (QCD), always something impressive to drop into those dinner party conversations!

To understand why this theory is so called, you should note that ‘chromo’ comes from the Greek word for ‘colour’. The interactions between quarks and gluons are described in terms of a new property of matter that is called \sphinxstylestrong{colour charge}, by analogy with conventional electric charge.

Just as electromagnetic interactions result from forces between electrically\sphinxhyphen{}charged particles, so strong interactions result from forces between colour\sphinxhyphen{}charged particles. However, whereas conventional electric charge comes in only one type that can either be positive or negative, colour charge comes in three types, each of which can be ‘positive’ or ‘negative’.

These three types of positive colour charge are known as red, green and blue, and their negatives are antired (or the colour cyan), antigreen (or the colour magenta) and antiblue (or the colour yellow). Confusingly, it is important to disregard your ideas of colour and to note that colour charge has nothing to do with colours of light, it is merely a naming convention.

Each quark can have any one of the three colour charges, and each antiquark can have any one of the three anticolour charges. So in effect, there are three versions of each type of quark: red up quarks, blue up quarks and green up quarks, for instance.

Gluons each carry a combination of colour and anticolour charge (such as red\textendash{}antiblue, blue\textendash{}antigreen or green\textendash{}antired), although they have zero electric charge. Leptons and photons do not have any colour charge associated with them. In all strong interactions there is therefore another conservation rule: colour charge is also conserved.


\paragraph{Question}
\label{\detokenize{content/session_00/Part_00_06:Question}}
A red up quark emits a red\sphinxhyphen{}antiblue gluon. What colour charge does the up quark now have?


\paragraph{Answer}
\label{\detokenize{content/session_00/Part_00_06:Answer}}
Since colour charge is conserved, the up quark must now have a blue colour charge.


\paragraph{Question}
\label{\detokenize{content/session_00/Part_00_06:id1}}
A green down quark absorbs a blue\sphinxhyphen{}antigreen quark. What colour charge does the down quark now have?


\paragraph{Answer}
\label{\detokenize{content/session_00/Part_00_06:id2}}
Since colour charge is conserved, the down quark must now have a blue colour charge. This model helps to explain many phenomena, such as why the only possible hadrons are baryons (consisting of three quarks), antibaryons (consisting of three antiquarks) and mesons (consisting of one quark and one antiquark).

Each of these composite particles has a net colour charge of zero. Any baryon must contain one quark with a red colour charge, one quark with a green colour charge, and one quark with a blue colour charge. By analogy with conventional colours: red + green + blue = white, a neutral colour with a net colour charge of zero. Likewise, antibaryons must contain one antiquark with an antired colour charge, one antiquark with an antigreen colour charge, and one antiquark with an antiblue colour charge.
Again, this gives a net colour charge of zero.

\sphinxincludegraphics[width=512\sphinxpxdimen,height=231\sphinxpxdimen]{{sm123_t06_p03_f06.tif}.jpg}

\sphinxstylestrong{Figure 7} (a) Three colour charges combine to produce a net colour charge of zero (i.e. white). (b) Three anticolour charges combine to produce a net colour charge of zero (i.e. white).

Similarly, the quark\textendash{}antiquark pairs that constitute a meson must have the opposite colour charge to each other: red + antired (cyan) = white for instance, which is a net colour charge of zero again.

Only particles with a net colour charge of zero are allowed to exist in an independent state, and this explains why single quarks and antiquarks are not seen in isolation. The locking up of quarks inside hadrons is referred to as \sphinxstylestrong{confinement}. Gluons do not have a net colour charge of zero either, so they too do not escape from strong interactions. Instead, gluons will decay into quark\textendash{}antiquark pairs, which in turn create further hadrons.


\subsection{7  Beta\sphinxhyphen{}decay at the level of quarks and leptons}
\label{\detokenize{content/session_00/Part_00_07:7_xa0_xa0Beta-decay-at-the-level-of-quarks-and-leptons}}\label{\detokenize{content/session_00/Part_00_07::doc}}
To conclude your exploration of the particle world, you will now examine more closely the radioactive processes in which particles change flavour, as this will lead you to the final type of fundamental interaction in which particles participate.

A particular type of radioactivity is known as beta\sphinxhyphen{}decay, which occurs in three forms known as beta\sphinxhyphen{}minus decay, beta\sphinxhyphen{}plus decay and electron capture. In each form, protons convert into neutrons, or vice\sphinxhyphen{}versa. For instance, in beta\sphinxhyphen{}minus decay, one of the neutrons in a nucleus is converted into a proton in a process that may be written as:

\sphinxincludegraphics[width=229\sphinxpxdimen,height=33\sphinxpxdimen]{{sm123_t06_p03_6_1.eps}.jpg}

As a neutron has the quark composition (udd) and a proton has the quark composition (uud), at the level of individual quarks, a beta\sphinxhyphen{}minus decay must involve a down quark converting into an up quark as follows:

\sphinxincludegraphics[width=229\sphinxpxdimen,height=33\sphinxpxdimen]{{sm123_t06_p03_6_2.eps}.jpg}

This quark conversion therefore lies at the heart of all beta\sphinxhyphen{}minus decay processes. A quick check confirms that electric charge is conserved in this process. The charge on the left\sphinxhyphen{}hand side of the equation is −e3, while the sum of the charges of the particles on the right\sphinxhyphen{}hand side is +2e3−e+0=−e3 also.

Following the example above, try to work out what quarks are involved in the conversions below:


\subsubsection{Question}
\label{\detokenize{content/session_00/Part_00_07:Question}}
In beta\sphinxhyphen{}plus decay, a proton is converted into a neutron with the emission of a positron and an electron neutrino:

\sphinxincludegraphics[width=229\sphinxpxdimen,height=33\sphinxpxdimen]{{sm123_t06_p03_6_3.eps}.jpg}

What is the quark conversion process that lies at the heart of beta\sphinxhyphen{}plus decay? Confirm that electric charge is conserved.


\subsubsection{Answer}
\label{\detokenize{content/session_00/Part_00_07:Answer}}
At the level of individual quarks, this is:

\sphinxincludegraphics[width=229\sphinxpxdimen,height=33\sphinxpxdimen]{{sm123_t06_p03_6_4.eps}.jpg}

The electric charge on the left\sphinxhyphen{}hand side is +2e3 while that on the right\sphinxhyphen{}hand side is − e3+e+0=+2e3, as expected.


\subsubsection{Question}
\label{\detokenize{content/session_00/Part_00_07:id1}}
In electron capture, a proton is converted into a neutron when it captures an electron, and subsequently emits an electron neutrino:

\sphinxincludegraphics[width=229\sphinxpxdimen,height=33\sphinxpxdimen]{{sm123_t06_p03_6_5.eps}.jpg}

Write down the quark conversion process that lies at the heart of electron capture. Confirm that electric charge is conserved.


\subsubsection{Answer}
\label{\detokenize{content/session_00/Part_00_07:id2}}
At the level of individual quarks, this is:

\sphinxincludegraphics[width=229\sphinxpxdimen,height=33\sphinxpxdimen]{{sm123_t06_p03_6_6.eps}.jpg}

The electric charge on the left hand side is +2e3−e=−e3, while that on the right hand side is −e3+0=−e3 as expected. To end this course, you will consider the explanation for these flavour\sphinxhyphen{}changing reactions in terms of the final fundamental interaction found in nature, known as the weak interaction.


\subsection{8  Weak interactions}
\label{\detokenize{content/session_00/Part_00_08:8_xa0_xa0Weak-interactions}}\label{\detokenize{content/session_00/Part_00_08::doc}}
\_\_Weak interaction\_\_s manifest themselves as reactions, or decays, in which some particles may disappear, while others appear. There is no structure that is bound together by a ‘weak force’, but weak interactions are vital for understanding the world around us.

Weak interactions were involved in most of the reactions in the very early Universe by which particles changed from one sort to another. They are therefore largely responsible for the overall mixture of particles from which the current Universe is made.

The most common example of a weak interaction is beta\sphinxhyphen{}decay and, as you saw earlier, there are three related processes, each of which is a different type of beta\sphinxhyphen{}decay. In each of these three processes the nucleus involved will change from one type of element to another, as a result of either increasing or decreasing its proton content by one. Each process relies on the weak interaction.


\subsubsection{8.1 W and Z bosons}
\label{\detokenize{content/session_00/Part_00_08:8.1_xa0W-and-Z-bosons}}
In the same way that photons and gluons are the quanta involved in electromagnetic and strong interactions, respectively, weak interactions involve other quanta ­\textendash{} known as \_\_W boson\_\_s and \_\_Z boson\_\_s.

In fact, there are two types of W boson, one with negative electric charge, the W− boson, and one with positive electric charge, the W+ boson. The two (charged) W bosons each have a mass of about 80 GeV/\sphinxstyleemphasis{c}2 whereas the (neutral) Z boson has a mass of about 90 GeV/\sphinxstyleemphasis{c}2. In weak interactions, W and Z bosons interact with each other, as well as with all quarks and leptons. The Universe would be an impossibly boring place without them.

As you know, the beta\sphinxhyphen{}minus decay of a nucleus occurs when a neutron turns into a proton, with the emission of an electron and an electron antineutrino. At most, a few MeV of energy are released in this process, corresponding to the difference in mass between the original nucleus and the resultant nucleus. At the quark level, the explanation is that a down quark, d, with a negative electric charge equal to one\sphinxhyphen{}third that of an electron is transformed into an up quark, u, with a positive electric
charge equal to two\sphinxhyphen{}thirds that of a proton.

A W− boson is emitted with one unit of negative electric charge, so conserving electric charge in the process. The mass energy of the W− boson is about 80 GeV, so it cannot possibly emerge from the nucleus as there are only a few MeV of energy available. In accordance with the energy\textendash{}time uncertainty principle it therefore rapidly decays to produce an electron and an electron antineutrino, setting the energy accounts straight.

\sphinxincludegraphics[width=249\sphinxpxdimen,height=251\sphinxpxdimen]{{sm123_t06_p03_f07.eps}.jpg}

\sphinxstylestrong{Figure 8} A beta\sphinxhyphen{}minus decay process involves the creation and disappearance of a W− boson. A down quark decays into a W− boson and an up quark. The W− boson subsequently decays into an electron and an electron antineutrino.

In weak interactions, the total number of quarks minus the total number of antiquarks is the same both before and after the interaction. The number of leptons is also conserved. In the example of beta\sphinxhyphen{}minus decay, there are no leptons initially present, and after the interaction there is one lepton and one antilepton \textendash{} a net result of zero again.

This is the explanation for why neutrinos and antineutrinos are produced in beta\sphinxhyphen{}decays. If they were not, then the rule of lepton conservation would be violated. Notice also that the production of a charged lepton is always accompanied by the corresponding flavour of neutrino. In all weak interactions:
\begin{itemize}
\item {} 
electric charge is conserved

\item {} 
the number of quarks minus the number of antiquarks is conserved

\item {} 
the number of leptons minus the number of antileptons is conserved

\item {} 
flavour changing of quarks or leptons is allowed, as long as these three rules are obeyed.

\end{itemize}


\paragraph{Question}
\label{\detokenize{content/session_00/Part_00_08:Question}}
Following the example of beta\sphinxhyphen{}minus decay above, explain how beta\sphinxhyphen{}plus decay involves the creation and demise of a W+ boson. Check that electric charge is conserved, that the number of quarks minus the number of antiquarks is conserved, and that the number of leptons minus the number of antileptons is conserved.


\paragraph{Answer}
\label{\detokenize{content/session_00/Part_00_08:Answer}}
\sphinxincludegraphics[width=237\sphinxpxdimen,height=251\sphinxpxdimen]{{sm123_t06_p03_f08.eps}.jpg}

\sphinxstylestrong{Figure 9} A beta\sphinxhyphen{}plus decay process involves the creation and disappearance of a W+ boson. An up quark decays into a W+ boson and a down quark. The W+ boson subsequently decays into a positron and an electron neutrino.

The electric charge is initially that of an up quark (+23e). The products of the initial decay are a down quark with charge −13e, and a W+ boson with charge +\sphinxstyleemphasis{e}, so charge is conserved here. The W+  boson subsequently decays into a positron with charge +\sphinxstyleemphasis{e} and a neutral electron neutrino, so charge is again conserved.

There is one quark present both before and after the decay, so the total number of quarks minus the number of antiquarks is conserved and equal to one. There are no leptons present initially, but one lepton (the electron neutrino) and one antilepton (the positron) are present at the end. Therefore, the total number of leptons minus the number of antileptons is also conserved and equal to zero. The third of the quanta involved in weak interactions is the Z0 boson with zero electric charge. An
example of the type of reaction involving the Z0 boson is a collision between an electron and a positron. This can create a Z0 boson from the mass energy of the electron\textendash{}positron pair, which subsequently decays into a muon neutrino and a muon antineutrino pair. Notice that there is one lepton and one antilepton both before and after the interaction.

\sphinxincludegraphics[width=274\sphinxpxdimen,height=131\sphinxpxdimen]{{sm123_t06_p03_f09.eps}.jpg}

\sphinxstylestrong{Figure 10} An electron\textendash{}positron pair undergo annihilation, creating a Z0 boson which subsequently decays to create a muon neutrino and muon antineutrino pair.


\subsubsection{8.2 The survival of the neutron}
\label{\detokenize{content/session_00/Part_00_08:8.2_xa0The-survival-of-the-neutron}}
You will now return to the puzzle mentioned at the start of this course. Apart from hydrogen, nuclei made solely of protons cannot exist. Neutrons are necessary to make nuclei stable, so the neutron is vital to the Universe. Without it there would be only a single element, hydrogen, making chemistry extremely dull, as it would be limited to a single molecule, H2, with no one to study it!

The rules of strong interactions allow the construction of a neutron (udd) in the same manner as a proton (uud). Indeed, in the first moments of the Universe it is believed that protons and neutrons were created in equal numbers. Nowadays, however, the Universe as a whole contains only about one neutron for every seven protons, and the vast majority of those neutrons are locked up inside helium nuclei. Clearly then, at some stage, neutrons have ‘disappeared’ from the Universe. How has this
happened?

The mass energy of a free neutron is about 1.3 MeV larger than that of a free proton. This energy difference exceeds the mass energy of an electron (which is about 0.5 MeV) and means that free neutrons (i.e. neutrons not bound within atomic nuclei) can undergo beta\sphinxhyphen{}minus decay, with a half\sphinxhyphen{}life of about 10 minutes.

This is believed to be the mechanism by which the proportion of neutrons in the Universe decreased from one in every two hadrons soon after the Big Bang, to only around one in seven today. Once neutrons are incorporated into helium nuclei they are immune from beta\sphinxhyphen{}minus decay, as helium nuclei are stable.

Yet there is still a puzzle: if free neutrons can decay into protons, how did the neutrons form helium nuclei in time to avoid the fate of decay that affected them when they were free? The answer is, like much in life, that it was a question of timing. The temperature of the Universe had fallen to a value that allowed the formation of helium nuclei only a few minutes after the Big Bang. Because free neutrons survive for about 10 minutes before decaying, there were still plenty of them around at
this time, and all those that had not yet decayed into protons were rapidly bound up into helium nuclei.

But if free neutrons only survived for, say, one second, there would not have been many neutrons left to form nuclei a few minutes after the Big Bang. The vast majority of them would have long since decayed into protons.

The relatively long lifetime of a free neutron is due to the fact that weak interactions (such as beta\sphinxhyphen{}minus decay) truly are weak, and therefore occur only rarely at low energies.

So there is a vital condition for life in the Universe: weak interactions must be truly weak at low energies. If they were as strong as electromagnetic interactions at low energies, beta\sphinxhyphen{}minus decay processes would happen much more readily and the lifetime of a free neutron would be much shorter. As a result, the vast majority of the neutrons in the Universe would have decayed before it became possible for them to find safe havens in atomic nuclei, and there would have been no elements other
than hydrogen in the Universe.

The reason weak interactions are so weak results from the large masses of the W and Z bosons, which are each around 100 GeV/\sphinxstyleemphasis{c}2. In order for any weak interaction to occur, a W or Z boson must be created, but it is difficult to produce the massive W and Z bosons when the available energy is much less than this.

Your entire existence therefore relies on the large mass of the W boson! And yet it’s a particle that many people have not even heard of…


\subsection{Conclusion}
\label{\detokenize{content/session_00/Part_00_09:Conclusion}}\label{\detokenize{content/session_00/Part_00_09::doc}}
This free course, \sphinxstyleemphasis{Particle physics}, provided an introduction to studying particle physics and introduced you to the physics that operates at the smallest scales of the Universe.

Delving inside the atomic nucleus, you have seen that protons and neutrons are composed of fundamental particles called quarks. Although everything around you is made up of just first\sphinxhyphen{}generation quarks (up and down) and first\sphinxhyphen{}generation leptons (electrons and electron neutrinos), you have seen that there are six flavours of quark and six flavours of leptons in total. Finally, you have examined the strong and weak interactions which govern how these particles behave.
\begin{itemize}
\item {} 
What you have learned about leptons, quarks and hadrons: There are six flavours of lepton, the lightest of which are the electron and electron neutrino; there are six flavours of quark, the lightest of which are the up and down quarks. All leptons and quarks have corresponding antiparticles with the same mass but opposite electric charge and colour charge (in the case of quarks). Combinations of three quarks are called baryons; combinations of three antiquarks are called antibaryons;
combinations of a quark and an antiquark are called mesons. As examples, a proton has the quark composition ‘uud’; a neutron has the quark composition ‘udd’; pions are mesons composed of up and down quarks and antiquarks.

\item {} 
What you have learned about strong and weak interactions: The strong interaction binds quarks together inside nucleons, and binds nucleons together inside nuclei; all strong interactions involve gluons. Quarks and gluons each carry a colour charge; baryons, antibaryons and mesons are all colour\sphinxhyphen{}neutral.In strong interactions: energy, electric charge, colour charge and the number of quarks minus the number of antiquarks are all conserved. The weak interaction allows leptons and quarks to
change flavour; all weak interactions, such as beta\sphinxhyphen{}decay, involve W or Z bosons. In weak interactions, energy, electric charge, the number of quarks minus the number of antiquarks, and the number of leptons minus the number of antileptons are all conserved.

\end{itemize}

This OpenLearn course is an adapted extract from the Open University course \sphinxhref{http://www.open.ac.uk/courses/modules/sm123}{SM123 Physics and space}.


\section{Session 01}
\label{\detokenize{index:session-01}}

\subsection{Glossary}
\label{\detokenize{content/session_01/Part_01_glossary:Glossary}}\label{\detokenize{content/session_01/Part_01_glossary::doc}}
\sphinxstylestrong{antibaryon}: A subatomic particle composed of three antiquarks. The antimatter counterpart of a baryon. A type of hadron. Examples include the antiproton and antineutron. \sphinxstylestrong{antimatter}: Every type of matter particle has a corresponding type of antimatter particle which has the same mass but opposite other properties, such as electric charge. \sphinxstylestrong{baryon}: A subatomic particle composed of three quarks. The matter counterpart of an antibaryon. A type of hadron. Examples include the proton and
neutron. \sphinxstylestrong{bottom}: (b) A type of quark \sphinxhyphen{} a fundamental particle with electric charge \textendash{}e/3. (The bottom quark is sometimes alternatively referred to as ‘beauty’.) \sphinxstylestrong{charm}: (c) A type of quark \sphinxhyphen{} a fundamental particle with electric charge +2e/3. \sphinxstylestrong{colour charge}: A property possessed by quarks (and antiquarks) and gluons. It plays a role in QCD equivalent to that of electric charge in QED. Quarks can possess any one of red, green or blue colour charge. \sphinxstylestrong{confinement}: The process by which
quarks and antiquarks remain locked up inside hadrons. \sphinxstylestrong{down}: (d) A type of quark \sphinxhyphen{} a fundamental particle with electric charge \textendash{}e/3. One of the constituent particles of both the proton and the neutron. \sphinxstylestrong{electron}: (e\sphinxhyphen{}) One of the component particles from which an atom is made. Electrons have a negative electric charge, and they surround the atom’s positively charged nucleus. Electrons carry the electric current in metals. An electron is a fundamental particle (a lepton) with electric
charge \textendash{}\sphinxstyleemphasis{e} and mass energy about 500 keV. Electrons are produced in beta\sphinxhyphen{}minus decay processes. \sphinxstylestrong{electron neutrino}: (e) A fundamental particle (a lepton) with zero electric charge. It is produced in beta\sphinxhyphen{}plus decay along with a positron. Its antiparticle is the electron antineutrino. \sphinxstylestrong{electronvolt}: (eV) The unit of energy corresponding to the energy converted when an electron moves through a voltage difference of 1V. The electronvolt is abbreviated to eV and 1 eV = 1.602 × 10\sphinxhyphen{}19 J.
\sphinxstylestrong{flavour}: Somewhat whimsical name used to describe the different types of lepton (i.e. electron, electron neutrino, muon, muon neutrino, tauon, tauon neutrino) and the different types of quark (up, down, strange, charm, top, bottom). \sphinxstylestrong{generations}: (of fundamental particles) The electron and electron neutrino are referred to as first\sphinxhyphen{}generation leptons, whereas the up and down quarks are referred to as first\sphinxhyphen{}generation quarks. The muon, muon neutrino, strange and charm quarks belong to the
second generation; the tauon, tauon neutrino, bottom and top quarks belong to the second generation. \sphinxstylestrong{gluon}: The quantum of energy associated with the strong interaction. It plays a role in QCD analogous to that of photons in QED. Unlike photons, gluons themselves experience the strong interaction (photons do not experience the electromagnetic interaction). This is because gluons possess colour charge. (Photons do not possess conventional electric charge.) Consequently, gluons have a very
short range and are never observed in isolation. \sphinxstylestrong{hadron}: A composite particle composed of quarks and/or antiquarks. Baryons are hadrons consisting of three quarks. \sphinxstylestrong{meson}: A subatomic particle composed of a quark and an antiquark. A type of hadron. Examples of mesons include the pions. \sphinxstylestrong{muon}: (μ\sphinxhyphen{}) A fundamental particle (a lepton) with electric charge \textendash{}\sphinxstyleemphasis{e} which is similar to an electron but with a mass about 200 times heavier. Its antiparticle is called the antimuon (µ+) \sphinxstylestrong{muon
neutrino}: () A fundamental particle (a lepton) with zero electric charge. Its corresponding antimatter particle is called the muon antineutrino. \sphinxstylestrong{neutron}: (n) One of the two types of particle found in the nucleus of an atom. The neutron has zero electric charge and a mass very close to that of the proton. As with the proton therefore, the relative atomic mass of a neutron is very close to unity on the scale of relative atomic masses. Neutrons are baryons, with the quark composition (udd).
They have an electric charge of zero, and a mass energy of about 1 GeV. \sphinxstylestrong{photon}: A particle of light or other electromagnetic radiation. Monochromatic light consists of photons that each have exactly the same amount of energy, called a quantum of energy. Therefore also, the quantum of energy associated with the electromagnetic interaction. Photons have no mass or electric charge and do not experience the electromagnetic interaction. (Compare this behaviour with gravitons and gluons, which
also have no mass or electric charge. \sphinxstylestrong{positron}: (e+) (antielectron) The antimatter counterpart to the electron. It has an electric charge of +\sphinxstyleemphasis{e}, but the same mass as the electron. It is produced in beta\sphinxhyphen{}plus decay processes. \sphinxstylestrong{proton}: (p) The nucleus of the hydrogen atom and one of two constituents of all other nuclei; the particle has a relative mass very close to unity on the scale of relative atomic masses. Protons are baryons, with the quark composition (uud). They have an electric
charge of +\sphinxstyleemphasis{e} and a mass energy of about 1 GeV. \sphinxstylestrong{quantum chromodynamics}: (QCD) The theory that describes the strong interaction. It explains these interactions as arising due to the exchange of gluons between particles that possess colour charge. Compare with quantum electrodynamics. \sphinxstylestrong{strange}: (s) A type of quark \sphinxhyphen{} a fundamental particle with electric charge \textendash{}\sphinxstyleemphasis{e}/3. \sphinxstylestrong{strong interaction}: The fundamental interaction between quarks and antiquarks responsible for binding them together as
triplets inside baryons and antibaryons, or as quark\textendash{}antiquark pairs inside mesons. A residual effect of the strong interaction is responsible for binding protons and neutrons together as nuclei. One of the four fundamental interactions, described by the modern theory of QCD. It describes interactions between particles that possess colour charge in terms of the exchange of quanta called gluons. Strong interactions get weaker with increasing energy of interaction. Compare with electromagnetic
interaction. \sphinxstylestrong{tauon}: (\sphinxhyphen{}) A fundamental particle (a lepton) with electric charge \textendash{}\sphinxstyleemphasis{e} which is similar to an electron but with a mass about 3500 times heavier. Its antiparticle is called the antitauon (+). \sphinxstylestrong{tauon neutrino}: () A fundamental particle (a lepton) with zero electric charge. Its antiparticle is called the tauon antineutrino. \sphinxstylestrong{top}: (t) A type of quark \sphinxhyphen{} a fundamental particle with electric charge +2\sphinxstyleemphasis{e}/3. (The top quark is sometimes alternatively referred to as ‘truth’.)
\sphinxstylestrong{up}: (u) A type of quark \sphinxhyphen{} a fundamental particle with electric charge +2\sphinxstyleemphasis{e}/3. One of the constituent particles of both the proton and the neutron. \sphinxstylestrong{virtual photon}: A photon involved in carrying the electromagnetic force from one part of an interaction to another. Virtual photons do not escape from processes and are never detectable directly. \sphinxstylestrong{W boson}: One of the quanta associated with the weak interaction. W bosons come in two varieties labelled W+ and W\sphinxhyphen{}, where the superscripts
denote the electric charge of the particles. Both W bosons have mass energies of about 80 GeV. \sphinxstylestrong{weak interaction}: Fundamental interaction involving both quarks and leptons. One of the four fundamental interactions. It describes interactions between particles in terms of the exchange of quanta called W bosons and Z bosons. There is no structure that’s bound together by a ‘weak force’. Weak interactions are responsible for processes such as beta\sphinxhyphen{}decay in which quarks change flavour and
lepton\textendash{}antilepton pairs are produced. \sphinxstylestrong{Z boson}: One of the quanta associated with the weak interaction. Z bosons are often labelled Z0, where the superscript denotes that it has zero electric charge. Z bosons have mass energies of about 90 GeV.



\renewcommand{\indexname}{Index}
\printindex
\end{document}